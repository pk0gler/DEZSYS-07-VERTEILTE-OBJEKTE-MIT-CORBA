%!TEX root=../document.tex
\section{Corba Allgemein (Overview)}
\label{sec:Corba Allgemein (Overview)}


	\subsection{Was ist CORBA ?}

	CORBA (\textbf{C}ommon \textbf{O}bject \textbf{R}equest \textbf{B}roker \textbf{A}rchitecture) vereinfacht das Erstellen verteilter Anwendungen und soll das Aufrufen externer Methoden ermöglichen bzw. vereinfachen.
	
	Der große Vorteil von CORBA ist die Plattformunabhängigkeit und ist somit, im Gegensatz zu anderen Umsetzungen, nicht an eine Spezielle Umgebung gebunden.
	CORBA setzt darauf, dass die Hersteller bzw. Communities, auf der Grundlage der Spezifikation eigene Object-Request-Broker Implementierungen erstellen und weiter verbessern. Deshalb können Hersteller Ihre Implementierungen für mehrere Programmiersprachen und auch unterschiedliche Betriebssysteme anbieten.
	
	Die gemeinsame Spezifikation ermöglicht dann die Kommunikation von Anwendungen untereinande, die mit unterschiedlichen Programmiersprachen erstellt worden sind, verschiedene ORBs nutzen und auf verschiedenen Betriebssystemen und Hardwareumgebungen laufen können.

	\subsection{Wie funktioniert CORBA ?}
	
	Mithilfe von der \textit{Interface Definition Language (IDL)} können formale Spezifikationen der Schnittstellen (Datentypen und Methodensignaturen), die ein Objekt für remote oder lokale Zugriffe zur Verfügung stellen, umgesetzt werden.
	
	Damit das ganze Prinzip funktioniert müssen diese definierten Schnittstellen (in Java Interfaces) für alle anderen Programmiersprachen umgesetzt werden.
	Dafür müssen diese nun von dem entsprechenden IDL-Compiler in äquivalente Beschreibungen der Schnittstellen kompiliert werden.
	
	Ebenfalls wird Quellcode, welcher zu der benutzten ORB-Implementierung passt, erstellt. Deiser Quelltext enthält, wie wir bereits von \textit{Remote Method Invocation} kennen, die Implementierung des Skelletons bzw. Stubs für Callback am Client usw. Durch dieses \textit{Vermittler-Pattern} erscheinen remote Aufrufe fast so einfach wie lokale Aufrufe und verbirgt somit die Komplexität der damit verbundenen Netzwerkkommunikation.

\section{Ergebnisse}
\label{sec:Ergebnisse}

	\subsection{Vorbereitung (Installieren und Konfigurieren)}

	\subsection{Tabelle}
	\renewcommand{\arraystretch}{1.5}
	\begin{table}[!h]
		\center
		\begin{tabular}{ | @{\hspace{3mm}} c @{\hspace{3mm}} | @{\hspace{3mm}} l @{\hspace{3mm}} | }
			\hline Header & Kopf\\ \hline\hline
			\textbf{Lorem} & Ipsum dolor sit amet, consetetur sadipscing elitr\\ \hline
			\textbf{Ipsum} & At vero eos et accusam et justo duo dolores et ea rebum.\\
				& Stet clita kasd gubergren, no sea takimata sanctus\\ \hline
			\textbf{Dolor} & Consetetur sadipscing elitr, sed diam nonumy\\\hline
		\end{tabular}
		\caption{Lorem ipsum dolor sit amet \cite{tanenbaum2007verteilte}}
		\label{methoden}
	\end{table}


	\subsection{Aufzählung}
	
	\begin{itemize}
		\item \textbf{Lorem ipsum:} dolor sit amet, consetetur sadipscing elitr
		\item sed diam nonumy eirmod tempor invidunt ut labore et dolore magna aliquyam erat
		\item ut labore et dolore magna aliquyam erat, sed diam voluptua
	\end{itemize}
	
	
	\subsection{Code}

	At vero eos et accusam et justo duo dolores et ea rebum.
	
	\begin{lstlisting}[style=C++, caption=Implizite Transaktion \cite{tanenbaum2007verteilte}]
	try{
	   gTransCur.begin();
	   //Perform the operation inside the transaction
	   not_registered = 
	       gRegistrarObjRef.register_for_courses(student_id,selected_course_numbers);
	
	
	   if (not_registered != null)
	
	     //If operation executes with no errors, commit the transaction
	     boolean report_heuristics = true;
	     gTransCur.commit(report_heuristics);
	
	   } else gTransCur.rollback();
	
	
	} catch(org.omg.CosTransactions.NoTransaction nte) {
	    System.err.println("NoTransaction: " + nte);
	    System.exit(1);
	} catch(org.omg.CosTransactions.SubtransactionsUnavailable e) {
	    System.err.println("Subtransactions Unavailable: " + e);
	    System.exit(1);
	} catch(org.omg.CosTransactions.HeuristicHazard e) {
	    System.err.println("HeuristicHazard: " + e);
	    System.exit(1);
	} catch(org.omg.CosTransactions.HeuristicMixed e) {
	    System.err.println("HeuristicMixed: " + e);
	    System.exit(1);
	}
	\end{lstlisting}
