%!TEX root=../document.tex

\section{Einführung}
Verteilte Objekte haben bestimmte Grunderfordernisse, die mittels implementierten Middlewares leicht verwendet werden können. Das Verständnis hinter diesen Mechanismen ist aber notwendig, um funktionale Anforderungen entsprechend sicher und stabil implementieren zu können.

\subsection{Ziele}
Diese Übung gibt eine einfache Einführung in die Verwendung von verteilten Objekten mittels CORBA. Es wird speziell Augenmerk auf die Referenzverwaltung sowie Serialisierung von Objekten gelegt. Es soll dabei eine einfache verteilte Applikation in zwei unterschiedlichen Programmiersprachen implementiert werden.

\subsection{Voraussetzungen}
\begin{itemize}
	\item Grundlagen Java, C++ oder anderen objektorientierten Programmiersprachen
	\item Grundlagen zu verteilten Systemen und Netzwerkverbindungen
	\item Grundlegendes Verständnis von nebenläufigen Prozessen
\end{itemize}

\subsection{Aufgabenstellung}
Verwenden Sie das Paket ORBacus oder omniORB bzw. JacORB um Java und C++ \linebreak ORB-Implementationen zum Laufen zu bringen.

Passen Sie eines der Demoprogramme (nicht Echo/HalloWelt) so an, dass Sie einen Namingservice verwenden, welches ein Objekt anbietet, das von jeweils einer anderen Sprache (Java/C++) verteilt angesprochen wird. Beachten Sie dabei, dass eine IDL-Implementierung vorhanden ist um die unterschiedlichen Sprachen abgleichen zu können.

Vorschlag: Verwenden Sie für die Implementierungsumgebung eine Linux-Distribution, da eine optionale Kompilierung einfacher zu konfigurieren ist.
\clearpage
